\documentclass{article}
\usepackage{graphicx} % Required for inserting images
\usepackage{makecell}

\title{CSE 415 Assignment 2 Report: \\
Evaluating Search Algorithms and Heuristics}
\author{Jah Chen, Eugene Cheung}
\date{January 2025}

\begin{document}

\maketitle

\section{Introduction}
We are Jah Chen and Eugene Cheung.
This is our report for Assignment 2 covering both blind search algorithms
and heuristic search.

\section{Report on Part A: Problem Formulation and Blind Search Algorithms}

\subsection{Part A Step 4 (c)}

(The 4th if-statement of can_move function states that if there's at least one human remaining on the on the ferry's current side after taking 
a group of humans and robots onto the ferry and that the those remaining humans are outnumbered by the robots, it's illegal to move the ferry.)


\subsection{Part A Step 8}

(your answer for the question in Part A step 8 goes into the table below, as
well as the path details in 2.3 and explanations in 2.4.)

The paths are not required in the report for the entries marked "skip."

{\flushleft
\begin{tabular}{|l|p{2cm}|p{2cm}|p{3cm}|}
\hline
\parbox{3.5cm}{Problem and\\ Algorithm} & Path Found & Path length & \#Nodes Expanded \\
\hline
\makecell[l]{Humans, Robots\\ and Ferry / DFS} & (skip) & 9 & 10\\
\hline
\makecell[l]{Humans, Robots\\ and Ferry / BreadthFS} & & 7 & 10 \\
\hline
\makecell[l]{Farmer, Fox, Chicken\\ and Grain/ DFS} & & & \\
\hline
\makecell[l]{Farmer, Fox, Chicken\\ and Grain/ BreadthFS}  & & 7 & 24\\
\hline
\makecell[l]{4-Disk Towers of\\ Hanoi/DFS} & (skip) & 40 & 40 \\
\hline
\makecell[l]{4-Disk Towers of\\ Hanoi/BreadthFS} & & 15 & 70 \\
\hline
\end{tabular}}

\subsection{Part A Step 8, Path details}
 Paths found (if not shown in the table).  Copy the state sequences
 obtained from the search algorithm on the requested problems.

 \begin{itemize}
 \item HRF/BreadthFS:
 Solution path:

 H on left:3
 R on left:3
   H on right:0
   R on right:0
 ferry is on the left.


 H on left:2
 R on left:2
   H on right:1
   R on right:1
 ferry is on the right.


 H on left:3
 R on left:2
   H on right:0
   R on right:1
 ferry is on the left.


 H on left:0
 R on left:2
   H on right:3
   R on right:1
 ferry is on the right.


 H on left:2
 R on left:2
   H on right:1
   R on right:1
 ferry is on the left.


 H on left:0
 R on left:1
   H on right:3
   R on right:2
 ferry is on the right.


 H on left:1
 R on left:1
   H on right:2
   R on right:2
 ferry is on the left.


 H on left:0
 R on left:0
   H on right:3
   R on right:3
 ferry is on the right.

Path length: 7
Nodes expanded: 10
Max open list length: 2
 \item FFCG/DFS: 
 \item FFCG/BreadthFS:

 Left Bank: ['farmer', 'fox', 'chicken', 'grain']
Right Bank: []
Boat is on the left bank.

Left Bank: ['fox', 'grain']
Right Bank: ['farmer', 'chicken']
Boat is on the right bank.

Left Bank: ['fox', 'grain', 'farmer']
Right Bank: ['chicken']
Boat is on the left bank.

Left Bank: ['grain']
Right Bank: ['chicken', 'farmer', 'fox']
Boat is on the right bank.

Left Bank: ['grain', 'farmer', 'chicken']
Right Bank: ['fox']
Boat is on the left bank.

Left Bank: ['chicken']
Right Bank: ['fox', 'farmer', 'grain']
Boat is on the right bank.

Left Bank: ['chicken', 'farmer']
Right Bank: ['fox', 'grain']
Boat is on the left bank.

Left Bank: []
Right Bank: ['fox', 'grain', 'farmer', 'chicken']
Boat is on the right bank.
Path length: 7
Nodes expanded: 24
Max open list length: 7

 \item 4-Disk TOH/BreadthFS:

 [[4, 3, 2, 1] ,[] ,[]]
[[4, 3, 2] ,[1] ,[]]
[[4, 3] ,[1] ,[2]]
[[4, 3] ,[] ,[2, 1]]
[[4] ,[3] ,[2, 1]]
[[4, 1] ,[3] ,[2]]
[[4, 1] ,[3, 2] ,[]]
[[4] ,[3, 2, 1] ,[]]
[[] ,[3, 2, 1] ,[4]]
[[] ,[3, 2] ,[4, 1]]
[[2] ,[3] ,[4, 1]]
[[2, 1] ,[3] ,[4]]
[[2, 1] ,[] ,[4, 3]]
[[2] ,[1] ,[4, 3]]
[[] ,[1] ,[4, 3, 2]]
[[] ,[] ,[4, 3, 2, 1]]
Path length: 15
Nodes expanded: 70
Max open list length: 16

 \end{itemize}

 \subsection{Part A Step 8,  Explanations of Certain Differences, Using Towers-of-Hanoi  }

\begin{paragraph}
(i. Why the maximum length of the OPEN list is more for one algorithm
than the other)
In BFS, the OPEN list grows as it explores nodes level by level. All possible states at the current level must be stored before moving to the next level.
Since BFS systematically explores all neighbors, the number of states stored in the OPEN list is higher. This explains why the maximum OPEN list length is 16 for BFS in this case.

\end{paragraph}
\begin{paragraph}
(ii. Why why the solution PATH length is different for one algorithm from that of the other. )
DFS explores paths deeply before backtracking, meaning it doesn’t need to store as many intermediate states in memory. Once a path is finished, it backtracks and removes unnecessary states from the OPEN list.
DFS is memory-efficient and typically requires less space. Here, the maximum length of the OPEN list is 7, which is much smaller than BFS.
\end{paragraph}

% -----------------------------
\newpage
\section{Report on Part B: Heuristics for the Eight Puzzle}

(Your results for Part B should be reported in the table below.)


\subsection{Results with Heuristics for the Eight Puzzle}

{\flushleft
\begin{tabular}{|c|l|c|l|c|c|c|}
\hline
Puzzle & Heuristic & Solved? & \# Soln Edges & Soln Cost & \# Expanded & Max Open\\
\hline
A & none (UCS) & Y & & & & \\
\hline
A & Hamming & Y & & & & \\
\hline
A & Manhattan & Y & & & & \\
\hline
B & none (UCS) & & & & & \\
\hline
B & Hamming &  & & & & \\
\hline
B & Manhattan & Y & & & & \\
\hline
C & none (UCS) & & & & & \\
\hline
C & Hamming &  & & & & \\
\hline
C & Manhattan & Y & & & & \\
\hline
D & none (UCS) & & & & & \\
\hline
D & Hamming &  & & & & \\
\hline
D & Manhattan & Y & & & & \\
\hline

\end{tabular} }

\begin{verbatim}
Puzzle A: [3,0,1,6,4,2,7,8,5]
Puzzle B: [3,1,2,6,8,7,5,4,0]
Puzzle C: [4,5,0,1,2,8,3,7,6]
Puzzle D: [0,8,2,1,7,4,3,6,5]
\end{verbatim}

\subsection{(Optional) Evaluating Our Custom Heuristics}

Describe your custom heuristic here.  What is the underlying intuition for it?
Is it admissible? Why or why not, or why is it difficult to determine if that
is the case.  How would you compare its computational cost with that of
the Hamming heuristic and the Manhattan distance heuristic?

What puzzles did you try it on, and how did it compare?
You may add rows to your table above to support your answer about comparison.
(Give your heuristic an appropriate short name to identify it in the table.)

\newpage
\section{Partnership Retrospective}

\subsection{Partnership?}
Did you work in a partnership? (yes or no).

If so, who were the partners (repeating your names from below the title on the first page)?

\subsection{Collaboration}
Also if so, how did you did you divide the work of this assignment?

\subsection{Newness of the Collaboration}
If this was a new sort of experience for either of you, please mention that,
and in what way(s) it felt new.




\end{document}
